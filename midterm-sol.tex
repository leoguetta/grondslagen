\documentclass[a4paper, 10pt]{article}
\usepackage[utf8]{inputenc}
\usepackage[T1]{fontenc}
\usepackage{lmodern}
\usepackage[english]{babel}

\usepackage[
  a4paper,
  left=1.5cm,
  right=1.5cm,
  top=1cm,
  bottom=2cm
  ]{geometry}
  
\usepackage{amsmath}
\usepackage{amssymb} 
\usepackage{amsthm} 
\usepackage{thmtools}
\usepackage{enumitem}

\newcommand{\R}{\mathbb{R}}
\newcommand{\N}{\mathbb{N}}

\title{Midterm Exam: Grondslagen van de Wiskunde}
\date{December 16, 2025, 09:00-12:00}

\begin{document}
\maketitle

\noindent \textsc{This exam consists of 5 exercises; see also the
  back side.}

\noindent Every exercise is worth 10 points; your grade is the total
divided by 5. If an exercise contains several questions, the number of
points attributed to each question is indicated.

\noindent As the exam is \textbf{closed book}, some basic definitions
about posets, which you may use freely, are recalled below.

\noindent Advice: start first with the exercises that you can do, and
then think about the rest. Succes!

\medskip

\hrule

\medskip

\noindent\textit{A few definitions about posets:} Recall that a poset is a pair
$(X,\leq)$ of a set $X$ and a binary relation $\leq$ such that:
\begin{itemize}
\item (reflexivity) $x\leq x$ for any $x\in X$
\item (anti-symmetry) $x\leq y$ and $y \leq x$ imply $x = y$ for any
  $x,y \in X$, 
\item (transitivity) $x\leq y$ and $y \leq z$ imply $x\leq z$, for any
  $x,y,z \in X$.
\end{itemize}
 We say furthermore that the order is \emph{linear} (or \emph{total}) if
for any $x,y \in X$, we have $x \leq y$ or $y \leq x$.

\noindent We will often only refer to a poset as $X$ instead of
$(X,\leq)$.

\noindent A poset is \emph{well-ordered} if every non-empty subset has
a least element.

\noindent A function $f \colon L \to M$ between well-ordered sets is
an \emph{embedding} if it is strictly increasing (see Exercise 1) and
for any $x \in L$ and any $y\in M$ such that $y \leq f(x)$, there
exists $x' \leq x$ with $f(x')=y$.

\medskip

\hrule

\medskip 

\medskip

\noindent\textbf{Exercise 1.} 

  \begin{enumerate}
  \item (2 points) Let $P$ and $Q$ be posets and $f \colon P \to Q$ a
    function. Prove that if $P$ is linearly ordered, then the following
    assertions are equivalent:
    \begin{enumerate}[label=(\alph*)]
    \item $f$ is order-preserving (if $x \leq y$ then $f(x)\leq f(y)$)
      and injective,
    \item $f$ is strictly increasing: if $x<y$ then $f(x)<f(y)$,
    \item $f$ satisfies the condition: $x<y$ if and only if $f(x)<f(y)$.
    \end{enumerate}
  \item (4 points) Let $L$ and $M$ two well-ordered sets, $f \colon L \to M$ an
    embedding and $g \colon L \to M$ a strictly increasing
    function. Prove that for every $x \in L$, we have
    \[
      f(x)\leq g(x).
    \]
  \item (4 points) Let $L$ and $M$ be two well-ordered sets, prove that there
    exists an embedding $L \to M$ if and only if there exist a strictly
    increasing function $L \to M$.
  \end{enumerate}

  \medskip
  \noindent\textbf{Solution}
  \begin{enumerate}
  \item $(a) \Rightarrow (b)$: Suppose that $f$ is order-preserving and
  injective, and let $x<y$. Since $f$ is order preserving we have
  $f(x)\leq f(y)$, but since $f$ is injective we have $f(x)\neq f(y)$,
  hence $f(x)<f(y)$.

  $(b)\Rightarrow (a)$ Suppose that $f$ is strictly
  increasing. Clearly $f$ is order-preserving, hence we have to prove
  that $f$ is injective. Let $x,y \in P$ such that $f(x)=f(y)$. Since
  $P$ is linearly ordered, we either have $x>y$ or $x<y$ or
  $x=y$. Using that $f$ is strictly increasing, we see that the only
  possibility is $x=y$.
  
  $(b)\Rightarrow (c)$: Suppose that $f$ is strictly increasing and
  let $x,y \in P$ such that $f(x)<f(y)$. Suppose by contradiction that
  $x \nless y$. Because $P$ is linearly ordered, this means that necessarily $x
  \geq y$. Using that $f$ is increasing, this implies that
  $f(x)\geq f(y)$ which is a contradiction. Hence, we have $x<y$.

  $(c)\Rightarrow (b)$ trivial.
  \item Let $A \subseteq L$ be the set of those $x$ such that
    $f(x)\leq g(x)$. Since $f$ is an embedding, we have $f(0)=0$,
    hence $0 \in A$. Now let $x\neq 0 \in L$ such that $L_{<x} \subseteq A$,
    there are two cases:
    \begin{itemize}
      \item if $x$ is a successor, $x=y+1$. Because $g$ is strictly
        increasing, we have $g(y)<g(x)$, hence $g(y)+1\leq g(x)$. Because $f$ is an
        embedding, we have $f(x)=f(y)+1$. By hypothesis, $y \in A$,
        hence $f(y)\leq g(y)$. All in all, we have
        \[
          f(x)=f(y)+1\leq g(y)+1 \leq g(x).
        \]
      \item if $x$ is limit, then $x=\sup\{y < x\}$. Because $g$ is
        strictly increasing, we have $g(y)<g(x)$ for all $y<x$. In
        particular, $\sup\{g(y)\,\vert\,y<x\}\leq g(x)$. Now because $f$
        is an embedding, we have $f(x)=\sup\{f(y)\,\vert\,y<x\}$,
        hence
        \[
          f(x)=\sup\{f(y)\,\vert\,y<x\}\leq \sup\{g(y)\,\vert\,y<x\}\leq g(x),
        \]
        where the first inequality comes from the fact that
        $L_{<x}\subseteq A$, hence for any $y < x$, we have $f(y)\leq g(y)$.
      \end{itemize}
    \item The ``only if'' part is trivial, since, by definition, an
      embedding is strictly increasing. Let us prove the ``if''
      part. Let $g \colon L \to M$ be a strictly increasing
      function. We are going to define an embedding $f \colon L \to M$
      be transfinite recursion. We set $f(0)=0$. Now, let $x \in L$,
      with $x \neq 0$, 
      such that we have define an embedding $f\vert_{L_{<x}} \colon
      L_{<x} \to M$, we need to give a rule to define $f(x)$. There
      are two cases:
      \begin{itemize}
        \item $x$ is a successor, $x=y+1$, then we have to define
          $f(x)=f(y)+1$, but to do that we need to be sure that $f(x)$
          is not a greatest element (and thus that the successor of
          $f(x)$ does exist). Applying question 2 of the exercise with
          $L_{<x}$ in place of $L$, we have
          that $f(y)\leq g(y)$, and because $g$ is strictly
          increasing, we have $g(y)<g(x)$, which proves that $f(y)$ is
          not a greatest element. 
        \item $x$ is limit, then $x=\sup\{y < x\}$, then we have to
          define $f(x)=\sup\{f(y)\,\vert\,y<x\}$, but to do that we
          need to be sure that this supremum exist, hence we need to
          show that the set $\{f(y)\,\vert\,y<x\}$ has an upper bound. Applying question 2 of the exercise with
          $L_{<x}$ in place of $L$, we have
          that $f(y)\leq g(y)$, for every $y<x$. Since, $g$ is
          strictly increasing, $g(x)$ is an upper bound of the set
          $\{g(y)\,\vert\,y<x\}$ and thus of the set $\{f(y)\,\vert\,y<x\}$.
      \end{itemize}
      It is immediate to check that the function $f$ such defined is
      an embedding.
    \end{enumerate}
    
  \medskip
  
\noindent\textbf{Exercise 2.} A \emph{tree} is a poset $(T,\leq)$ such
that:
\begin{enumerate}[label=(\roman*)]
\item it has a least element, called the \emph{root},
\item for every
  $x \in T$, the set $\{y \in T\,\vert\, y \leq x\}$ is well-ordered.
\end{enumerate}
Given an element $x \in T$, an \emph{immediate successor} of $x$ is an
element $y$ such that $x < y$ and there is no $z$ such that $x < z <
y$.
We say that a tree is \emph{infinite} if the underlying set is
infinite, and we say that it is \emph{locally finite} if for every
element $x \in T$, the set of immediate successors of $x$ is finite.
\begin{enumerate}
\item (3 points) Let $(T,\leq)$ be an infinite tree which is locally
  finite. Prove that if $x$ is an element such that the set  $\{y \in
  T\,\vert\, y \geq x\}$ is infinite, then there is an immediate
  successor of $x$ having the same property.
\item (7 points) Prove the famous K\H{o}nig's lemma: For every infinite and locally
  finite tree, there exists an infinite strictly increasing
  sequence
  \[
    x_0 < x_1 < x_2 < \dots.
  \]
  You have to make explicit where you use the axiom of choice in your answer.
\end{enumerate}

\medskip
\noindent\textbf{Solution}

\begin{enumerate}
\item Let us first prove the following fact: if $x<y$, then $x'\leq y$
  for some $x'$, immediate successor of $x$. For that, it suffices to
  notice that the set $\{z \leq y \,\vert\, z \in T\}$ is well-ordered
  by definition, hence, because $x$ is in that set and is not a greatest
  element, it has a successor $x+1$ in that set. It is then trivial to
  check that this element is an immediate successor of $x$ in $T$.

  From this fact it follows that for any $x \in T$, we have
  \[
    \{y \geq x \,\vert\, y \in T\} = \{x\} \cup \bigcup_{x'\in \mathrm{Succ}(x)}\{y \geq
    x' \,\vert\, y \in T\}.     
  \]
  Where $\mathrm{Succ}(x)$ is the set of all the immediate successors of
  $x$. By hypothesis, there are only a finite number of these, and
  since a finite union of finite sets is finite, if $ \{y \geq x
  \,\vert\, y \in T\}$ is infinite, then necessarily one of
  the set $\{y \geq
  x' \,\vert\, y \in T\}$ has to be infinite.
\item Let me give two possible solutions.
  \begin{itemize}
    \item A slightly
    informal one:
    Notice that if such a sequence exist, then for every $x_n$,
    then the set $\{y \geq x_n \,\vert\, y \in T\}$ is infinite. So,
    we are going to construct by recursion such a sequence with this
    property. We define $x_0$ to be the root, and if the sequence has be
    defined up to $x_n$ for some $n \in \N$, (with the property that
    $\{y \geq x_n \,\vert\, y \in T\}$ is infinite), then we define $x_{n+1}$ to
    be any of its immediate successor whose sets of greater elements
    is infinite. The existence of such an immediate successor follows
    from the previous question.

    The use of the axiom of (countable) choice here follows from the
    fact that at each step we have to make a choice, because there can
    be more than one immediate successor of $x_n$ having the desired
    property. In particular, we need to make a (countable) infinite
    number of choices, hence the use of the axiom of choice.
  \item A formal one using (AC): Let $A$ be the set of finite
    sequences $x_0 < x_1<\dots < x_n$ such that each $x_i \in T$ is such
    that the set $\{y \in T\,\vert\,y \geq x_i\}$ is infinite, and let
    $B=A\cup \{\ast\}$, where $\{\ast\}$ is disjoint from $A$. Let $f
    \colon A \to B$ be the function such that $f(x_0)=\ast$ and
    $f(x_0 <\dots < x_n)=(x_0< \dots <x_{n-1})$ for $n\geq 1$. By the
    previous question, the function $f$ is surjective. Hence, by the axiom
    of choice, it admits a section $s \colon B \to A$. For the sake of
    formality, let $h \colon A \to T$ be the function that ``picks the
    last element'', i.e.\ $h(x_0 <\dots <
    x_n)=x_n$. Then, we can
    define by recursion a sequence
    \[x_0 < x_1  < x_2  < \dots \]
    as $x_0=r$ the root, and $x_{n+1}=h(s(x_0 <\dots <
    x_n))$.
  % \item A formal one using Zorn's lemma: Let $S$ the subset of $T$ of
  %   these elements $x$ such that the set $\{y \in T\,\vert\,y \geq
  %   x\}$ is infinite, and let $X$ be the set of chains of $S$ ordered
  %   by inclusion. Let us show that we can apply Zorn's lemma on
  %   $X$. Clearly $X$ is non-empty. Now, let $C$ be a non-empty chain
  %   of $X$ (that is a chain of chains of $S$, ordered by
  %   inclusion!). Then consider $\bigcup C$, which is the subset of $S$
  %   of all these elements that are in one of the elements of $C$. Let
  %   us show that this subset, equipped with the order induced by $T$,
  %   is a chain, i.e.\ that it is linearly ordered. Let $x_0,x_1 \in C$. By
  %   definition, this means that there are two chains $C_0,C_1 \in C$
  %   such that $x_0 \in C_0$ and $x_1 \in C_1$. Because, $C$ is
  %   linearly ordered, we have either $C_0 \subset C_1$ or $C_1 \subset
  %   C_0$. Without loss of generality, let us suppose that we are in
  %   the first case. In particular, this means that $x_0 \in C_1$ as
  %   well. Because $C_1$ is a chain of $S$, we either have $x_0 \leq
  %   x_1$ or $x_1 \leq x_0$, which is exactly what we wanted to
  %   show. In particular, $\bigcup C$ is a upper bound of $C$ in
  %   $X$. Hence, by Zorn's lemma, we know that that the set $X$ admets
  %   a maximal element. Let $A$ be such a maximal element. By
  %   definition, it is a chain of $S$ (and hence of $T$). Notice that
  %   $A$ is necessarily infinite, otherwise it would have a greatest
  %   element $x$, and because $\{y \in T\,\vert\,y \geq
  %   x\}$ is infinite, we could extend this chain by adding another
  %   element greater than $x$, which would contradict the maximality of
  %   $A$.

  %   All in all, we get an infinite chain of $S$ (and thus of
  %   $T$). This contains an infinite countable subset (which is also a
  %   chain), hence a sequence
  %   \[
  %     x_0 < x_1 < \dots
  %   \]
  %   of $T$.
  \end{itemize}
  \end{enumerate}
  
\medskip

\noindent\textbf{Exercise 3.} Let $X$ be an infinite set and let $A
\subseteq X^X$ an \emph{infinite} set of functions $X \to X$. Let
$\overline{A}$ be the smallest subset of $X^X$ such that: $A \subseteq
\overline{A}$ and $\overline{A}$ is closed under composition, i.e. if
$f,g \in \overline{A}$, then $f\circ g \in \overline{A}$.

Show that $\vert \overline{A} \vert = \vert A \vert$.

\medskip

\noindent\textbf{Solution} Obviously, we have $\vert A \vert \leq
\vert \overline{A} \vert$. For the other inequality, notice that any element of $\overline{A}$
can be represented as a finite sequence $(f_1,f_2,\dots,f_n)$ of
elements of $A$. Hence, we have
\[
  \vert \overline{A} \vert \leq \sum_{n > 0}\vert A\vert^n.
\]
Because, $A$ is infinite we have $\vert A\vert ^n=\vert A\vert$ for any $n> 0$, and thus
\[
  \sum_{n > 0}\vert A\vert^n=\sum_{n > 0}\vert A \vert = \omega \times
  \vert A \vert \leq \vert A \vert \times \vert A \vert = \vert A \vert,
\]
where the last inequality and last equality both comes from the fact
that $A$ is infinite. Hence, we have $\vert \overline{A}\vert \leq
\vert A \vert$. By Cantor-Schröder-Bernstein, we conclude that $\vert
A \vert = \vert \overline{A} \vert$.

\medskip

\noindent\textbf{Exercise 4.} Consider the language $L=\{f\}$, with $f$
is a unary function symbol, and the $L$-structure $R$, whose
underlying set is $\R$, and $f^R(x)=x^2$.
\begin{enumerate}
\item (5 points) Using $L$-formulas, define in $R$ the
  numbers $-1, 0$ and $1$. That is, give $L$-formulas with one free
  variable such that the set of elements of $R$ for which the formula
  holds is $\{-1\}$ (resp.\ $\{0\}$, resp.\ $\{1\}$). 
\item (5 points) Find an $L$-formula $\phi(x)$ such that the subset of $x \in
  R$ for which $\phi(x)$ holds is the subset $\{x \in \R \,\vert\, x> 0\}$.
\end{enumerate}

\medskip

\noindent\textbf{Solution}
\begin{enumerate}
\item We can use the formulas
  \[
    \begin{aligned}
    \phi_{-1}(x)&\equiv& \left(f(f(x))=f(x)\right)\wedge \neg\left(f(x)= x\right),\\
    \phi_{0}(x)&\equiv& \forall y\left((f(y)=x)\leftrightarrow (y=x)\right),\\
    \phi_{1}(x)&\equiv& \left((\phi(x)=x) \wedge \neg
                        \phi_0(x)\right).
    \end{aligned}
  \]

\item We can use the formula
  \[
    \phi_{>0}(x)\equiv \exists y \left((f(x)=y)\wedge \neg \phi_{0}(y)\right).
  \]
\end{enumerate}
  
\noindent\textbf{Exercise 5.} Let $(X,\leq)$ be a poset. The goal of
the exercise is to prove that there always exists a linear order $\preceq$ on
$X$ that extends $\leq$ (i.e.\ $x \leq y$ implies $x\preceq y$).
\begin{enumerate}
\item (4 points) Prove the statement in the case that $X$ is finite.
\item (6 points) Prove the general case using the compactness theorem.
\end{enumerate}

\medskip

\noindent\textbf{Solution}
\begin{enumerate}
  \item We proceed by induction on the cardinality $n$ of $X$. If
    $n=0$, then the assertion is trivial. Suppose proven the statement
    for $n\geq 0$ and let $X$ be of cardinality $n+1$. Pick any $x \in X$.
    By hypothesis there exists a linear order $\preceq$ on $X-\{x\}$ 
    which extends $\leq$. Without loss of generality, we can denote
    the elements of $X-\{x\}$ by $x_1,x_2,\dots, x_n$ according to the
    linear order, i.e.\ with
    \[
      x_1 \prec x_2 \prec \dots \prec x_n.
    \]
    Now consider the set $A=\{y \in X-\{x\} \,\vert\,y\nless x \}$. We
    have $A
    \subseteq X-\{x\}$. Note that $y \in A$ means that either $y > x$
    or $x$ and $y$ are not comparable (neither $y >x$ nor $x <y$).

    If $A$ is empty, this means that for
    every element $y \neq x$ of $X$, we have $y < x$. In particular,
    we extend $\preceq$ from $X-\{x\}$ to $X$ by asserting that $x$ is
    the greatest element of $X$ for $\preceq$.
    It is straightforward to see that $\preceq$ on $X$ extends $\leq$,
    and clearly $\preceq$ is linear.

    If $A$ is not empty, then it has a lowest element $x_k$ (for
    $\preceq$). If $k=0$,
    then we extend $\preceq$ from $X-\{x\}$ to $X$ by asserting that $x$ is the least
    element of $X$ for $\preceq$, otherwise, we just ``insert'' $x$
    between $x_{k-1}$ and $x_{k}$:
    \[
      x_{k-1} \prec x \prec x_{k}.
    \]
    It is then straightforward to check that $\preceq$ extends $\leq$
    on $X$. Let us prove that it is linear. We only need to prove that
    $x$ and any $y \neq x$ are always comparable for $\preceq$. If
    they were already comparable for $\leq$, then there is nothing to
    prove. If not, then it means that $y \in A$. Since $x$ is
    (strictly) smaller
    for $\preceq$ than the least element of $A$, then necessarily $x
    \prec y$. 
  \item Consider the language $L_{X}$ which has one binary relation
    symbol $\preceq$ and a constant $c_{x}$ for every $x \in X$, and
    consider the theory $T_{X}$ which contains the theory of
    linearly ordered sets $T_{\mathrm{lin}}$ as well as the set of axioms
    \[
      \{c_x \preceq c_y \,\vert\, \text{ for all }x \leq y \in X\}\cup
      \{\neg(c_x=c_y)\,\vert\, \text{ for all }x \neq y \in X\}.
    \]
    If we can prove that $T_{X}$ has a model $M$, then the function $X
    \to M, x \mapsto c_x$ is injective and increasing. Hence, taking the
    order on $X$ induced by that of $M$ proves exactly what we
    want. By the compactness theorem, to prove that such a model
    exists, it suffices to prove that every finite subtheory $T'
    \subseteq T_{X}$ has a model. But for any such theory we have
    \[
      T' \subseteq T_{\mathrm{lin}}\cup \{c_x \preceq c_y \,\vert\,
      \text{ for a finite set of }x\leq y \text{ in } X\}\cup
      \{\neg(c_x=c_y)\,\vert\, \text{ for a finite set }x \neq y \in X\}.
    \]
    The answer to question 1 proves in particular that the theory on
    the right has a model, q.e.d.
\end{enumerate}
\end{document}

%%% Local Variables:
%%% mode: LaTeX
%%% TeX-master: t
%%% End:
